\section{Introduction} \label{sec:introduction}

Machine learning has been a heated research topic these days. With this amazing tool, we are now capable of predicting the price of the stock price based on history, doing the classifying by just inputing the pixels of images.

Supervised learning is one of the most important sections for machine learning. And Regression is probably the core of supervised learning. By firstly inputing in the computer some of the training data, the machine would be able to learn the characteristics of the dataset and make predictions.

Gaussian Process Regression(GPR) is a supervised learning regression method which are getting increasingly welcome in both the research field and the industry. 
In a GPR, we take advantage of the flexibility and simplicity of a Gaussian Process and implement it into a regression problem. \\

There are still countless unsolved problems in the field of GPR.
In this paper, we would comprehensively introduce the concept of GPR, including most of the notable works.
Specifically, we focus on some marvellous kernel choosing works.
We also implemented several experiments to quantitatively analyze the performance of different \emph{Mean Functions}, \emph{Likelihood Functions} and the \emph{Inference Methods}
We seek to provide a practical overview in the field of GPR.

The structure of this pape is as follows: 
In section \ref{sec:intro}, we provide a precise overview of a Gaussian Process Regression. 
In section \ref{sec:related}, we briefly listed some of the notable progress on GPR.
Section \ref{sec:autokernel} describes a great recent work on the methods for auto-construction of the kernels.
In section \ref{sec:experiment}, we conduct two experiments and use them to compare the performance of different methods and algorithms.
Conclusions are drawn in section \ref{sec:conclusion}.

